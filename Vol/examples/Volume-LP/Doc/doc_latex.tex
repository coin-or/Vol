\documentclass{article}

\setlength{\evensidemargin}{.25in}
\setlength{\oddsidemargin}{.25in}
\setlength{\textwidth}{6.0in}
%\setlength{\parindent}{0.25in}
\setlength{\topmargin}{-0in}
\setlength{\textheight}{8.0in}
%\newtheorem{theorem}{Theorem}
%\newtheorem{proposition}{Proposition}
%\newtheorem{lemma}{Lemma}
%\newtheorem{corollary}{Corollary}

%%%%%%%%%%%%%%%%%%%%%%%%%%%%%%%%%%%%%%%%%%%%%%%%%%%%%%%%%%%%%%%%%%%%%%%%%%%%
\begin{document}
\bibliographystyle{siam}

\title{\bf The Volume Algorithm for Combinatorial Linear Programs}
\author{Francisco Barahona}
\date{January 24, 2001}
\maketitle

%%%%%%%%%%%%%%%%%%%%%%%%%%%%%%%%%%%%%%%%%%%%%%%%%%%%%%%%%%%%%%%%%%%%%%%%%%%%
\section{Introduction}

Here we have an implementation of the Volume algorithm (VA) originally
presented in \cite{BA}. We focus on {\it Combinatorial Linear Programs},
these are linear programs where the matrix has 0, 1, -1 coefficients
and the variables are bounded between 0 and 1. The VA has been very
effective at producing fast approximate solutions to these LPs, see
\cite{BA2}.
As a first step, a new user should be able to run our code ``as is''.
The input should be an MPS file.
We hope to receive reports about bugs and/or
successful experiences.

\smallskip

Initially this directory contains the files: {\tt Makefile, volume.hpp,
volume.cpp, lp.hpp, lp.cpp, lp.par, data, doc.ps, lpc.h, lpc.cpp,
names.h, names.cpp}. On a Unix system one
should type
\begin{center} {\tt make} \end{center}

\noindent and then

\begin{center} {\tt bin/myprogram} \end{center}

\noindent Then the code would run and produce the files {\tt primal.txt}
and {\tt dual.txt} that contain approximate solutions to both the primal
and the dual problem.

\smallskip

We assume that we have an LP like
\begin{eqnarray}
&\min c x  \label{fp1} \\
&Ax \sim b \label{fp2} \\
&l \leq x \le u. \label{fp3} 
\end{eqnarray}

Let $\pi$ be a set of
Lagrange multipliers for constraints (\ref{fp2}). When we dualize 
them we obtain the {\it lagrangian problem}

\begin{eqnarray*}
L(u) & = & \min (c-\pi A) x + \pi b, \\
&&l \leq x \le u.
\end{eqnarray*}

We apply the VA to maximize $L(\cdot)$ and to produce a
dual vector $\bar \pi$, and
primal vector $\bar x$ that is an approximate solution of
(\ref{fp1})-(\ref{fp3}).

In what follows we describe the files {\tt lp.par} and
{\tt data}.


%%%%%%%%%%%%%%%%%%%%%%%%%%%%%%%%%%%%%%%%%%%%%%%%%%%%%%%%%%%%%%%%%%%%%%%%%%%%
\section{lp.par}

This file contains a set of parameters that control the algorithm and contain
information about the data. Each line has the format

{\tt keyword=value}

\noindent where {\tt keyword} should start in the first column. If we add any
other character in the first column, the line is ignored or considered as a
comment. The file looks as below

\bigskip
\begin{verbatim}
fdata=data.mps
*dualfile=dual.txt
dual_savefile=dual.txt
primal_savefile=primal.txt
h_iter=0
var_ub=1.0


printflag=3
printinvl=20
heurinvl=100000000

greentestinvl=2
yellowtestinvl=2
redtestinvl=10

lambdainit=0.1
alphainit=0.01
alphamin=0.0001
alphafactor=0.5
alphaint=80

maxsgriters=2000
primal_abs_precision=0.02
gap_abs_precision=0.
gap_rel_precision=0.01
granularity=0.

\end{verbatim}

The first group of parameters are specific to LP and the user should
define them. {\tt fdata} is the name of the file containing the data. {\tt
dualfile} is the name of a file containing an initial dual vector. If we add
an extra character at the beginning ({\tt *dualfile}) this line is ignored,
this means that no initial dual vector is given. {\tt dual\_savefile} is the
name of a file where we save the final dual vector. If this line is missing,
then the dual vector is not saved. {\tt primal\_savefile} is the name of a file
to save the primal solution, if this
line is missing, then this vector is not saved. {\tt h\_iter} is the number of
times that the heuristic is run after the VA has finished. We did not include 
a heuristic in this implementation. {\tt var\_ub} is an upper bound
for all primal variables, for 0-1 problems we set {\tt var\_ub=1}.

The remaining parameters are specific to the VA. {\tt printflag} controls the
level of output, it should be an integer between 0 and 5. {\tt printinvl=k}
means that we print algorithm information every {\tt k} iterations. {\tt
heurinvl=k} means that the primal heuristic is run every {\tt k} iterations.
{\tt greentestinvl=k} means that after {\tt k} consecutive green iterations
the value of $\lambda$ is multiplied by 2. {\tt yellowtestinvl=k} means that
after {\tt k} consecutive yellow iterations the value of $\lambda$ is
multiplied by 1.1. {\tt redtestinvl=k} means that after {\tt k} consecutive
red iterations the value of $\lambda$ is multiplied by 0.67. {\tt lambdainit}
is the initial value of $\lambda$. {\tt alphainit} is the initial value of
$\alpha$. {\tt alphafactor=f} and {\tt alphaint=k} mean that every {\tt k}
iterations we check if the objective function has increased by at least 1\%,
if not we multiply $\alpha$ by {\tt f}.

There are three termination criteria. First {\tt maxsgriter} is the maximum
number of iterations. The second terminating criterion is as follows. {\tt
primal\_abs\_precision} is the maximum primal violation to consider a primal
vector ``near-feasible''. Let {\tt gap\_rel\_precision=$g$}, let $z$ be the
value of the current dual solution, and $p$ be the value of a current
near-feasible primal solution. If $|z| > 0.0001$ and
$$
        { | z - p | \over | z | } < g,
$$
then the algorithms stops. Let {\tt gap\_abs\_precision=$f$}, if $|z| \le
0.0001$ and $| z - p | < f$ then we stop. Finally, let {\tt granularity=$k$},
and let $U$ be the value of the best heuristic integer solution found. Then if
$ U - z < k$ we stop. We did not include any heuristic in this 
implementation.


%%%%%%%%%%%%%%%%%%%%%%%%%%%%%%%%%%%%%%%%%%%%%%%%%%%%%%%%%%%%%%%%%%%%%%%%%%%%
\section{data}

This is an MPS file. If a different type of input should
be used, one has to change the code in {\tt reader.cpp}


%%%%%%%%%%%%%%%%%%%%%%%%%%%%%%%%%%%%%%%%%%%%%%%%%%%%%%%%%%%%%%%%%%%%%%%%%%%%



\bibliography{doc}

\end{document}


